% !TEX encoding = UTF-8 Unicode
%!TEX TS-program = xelatex

\chapter{结论与展望}

\section{全文总结}
乘着祖国大力发展载人航天事业的契机,本文以液体火箭纵向耦合振动的稳定性分析为主要研究内容,比较深入的阐述了POGO问题的研究对象、作用机理、求解方法及预防手段。全文研究工作的要点及创新点主要包括:
\begin{enumerate}[leftmargin=0pt, align=parleft, itemindent=2\parindent, labelsep=0pt, label=\roman*).]
	\item 较为全面的总结了国内、外液体火箭纵向耦合问题的研究进展,指出了传统的液体火箭POGO稳定性分析方法如矩阵法、单传法和临界阻尼法,在求解非对称、非线性复传递矩阵特征值时遇到的主要问题及方法局限性。
	\item 发展了一种基于矩阵法和有理分式拟合法的耦合系统快速特征值求解算法。该方法能够利用类结构化建模技术将包含超越函数的管路推进系统反馈力传递函数等效变换为与结构动力学方程一致的形式,通过求解矩阵特征值问题以快速、精确的确定耦合系统动力学稳定性。
	\item 利用虚质量法及带孔箱底精细有限元建模技术,提出了一套液体火箭带液贮箱的三维轴对称建模方法。发展后的三维带液贮箱模型不仅能够提供足够完整且合理的结构系统模态参与耦合计算,并且为管路系统提供了更为科学和精确的入口端边界条件。结合MSC.Nastran提供的传递函数TF卡建模工具,文章还给出了一种能够使得有理分式形式的反馈力传递函数参与结构系统耦合计算的商业软件整合技术,基本实现了耦合系统时变复特征值计算的模块化与自动化。
	\item 将液体火箭结构系统阻尼特性的识别和建模引入耦合系统稳定性分析,指出了适用于液体火箭POGO仿真的阻尼施加方式。通过调整贮箱干/湿面材料随时间变化的比例阻尼系数,成功的利用有限元模型模拟了实测贮箱阻尼结果。
	\item 通过比较不同类型的液体火箭在多种工况下的POGO稳定性分析结果,一方面揭示了耦合系统特征值与管路系统关键参数(如蓄压器容积等)之间的相互联系,另一方面还指出了将耦合系统动态传递特性分析纳入液体火箭POGO稳定性分析的重要性和必要性。参考现阶段国内液体火箭的设计水平与制造工艺,提出了一些分析及防治液体火箭POGO振动的基本思路,给出了液体火箭纵向耦合振动的理论分析框架和试验设计框架。
\end{enumerate}

\section{工作展望}
鉴于现有的理论基础和时间、经费及研究条件限制,本文的工作还存在许多不足及值得改进的地方:
\begin{enumerate}[leftmargin=0pt, align=parleft, itemindent=2\parindent, labelsep=0pt, label=\arabic*).]
	\item 管路推进系统只考虑了单组元的燃烧推力生成模型,尚没有考虑双贮箱情形的液体三维流动耦合作用;
	\item 一些管路系统元件的时变参数如泵的柔度与管件的当地声速,需要更为精确的实验测定;
	\item 耦合系统的结构阻尼时变参数需要进一步的理论分析及实验支持;
	\item 液体火箭结构系统仅对带液贮箱引入了三维有限元模型,其余部件依然采用集中参数弹簧-质量模型,需要进一步建立箭体结构其它部分的三维模型与贮箱进行匹配;
	\item 在对液体火箭管路系统进行建模时,未考虑蓄压器等元件的非线性效应;
	\item 未考虑液体火箭在结构失稳时由于非线性效应而产生的极限环现象;
	\item 程序、算法需要进一步完善和工程化。
\end{enumerate}

POGO振动本身是一个十分复杂的多领域、多系统、多参数的非线性时变问题。在与之配套的理论分析基础和实验测量数据都很不完备的今天,若期待祖国的载人航天事业能走的更远、更坚实,大量的后续工作还需要继往开来的科研工作者持之以恒的付出和实践。愿以此文与之共勉。

